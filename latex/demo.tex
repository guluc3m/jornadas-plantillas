\documentclass{beamer}

\usetheme{simple}

\usepackage{lmodern}
\usepackage[scale=2]{ccicons}

\usepackage[utf8]{inputenc} % No olvides utilizar UTF8, especialmente si tu presentación está en castellano

\def\edicion{ % Cambiar para las siguientes jornadas
XXVII
}

\title{Demo} % Modifica el título de la presentación a tu gusto
\subtitle{Basada en la demo de introducción a Beamer de ShareLaTeX.com y el tema Simple por Facundo Muñoz} % Incluye un subtítulo si lo ves necesario
\author{GUL} % Cambia el autor de la presentación.
\institute{\edicion Jornadas Técnicas del GUL} % Esto no lo cambies
\date{Noviembre 2016} % Esto tampoco

\begin{document}

{
    \usebackgroundtemplate{%
        \parbox[c][\paperheight][c]{\paperwidth}{
            \centering\tikz\node[opacity=0.3] {\includegraphics[height=7cm]{img/logo1.png}};
        }
    }
    \setbeamertemplate{footline}{}
    \begin{frame}
        \titlepage
    \end{frame}
}
\addtocounter{framenumber}{-1}

%\setwatermark{\includegraphics[height=8cm]{img/logo1.png}}

% ------------------------------------------------------------------------------------------
% La presentación empieza aquí. La primera diapositiva puede dejarse tal cual o sustituirse.
% ------------------------------------------------------------------------------------------

\begin{frame}
    \frametitle{Tabla de contenidos} % Puedes ponerle un título más chulo a la diapositiva con el índice de contenidos o dejar éste
    \tableofcontents
\end{frame}

% La demo empieza aquí
\section{Demo}

% Texto estándar
\subsection{Texto}

\begin{frame}
    \frametitle{Texto}
    Lorem ipsum dolor sit amet, consectetur adipisicing elit, sed do eiusmod tempor incididunt ut labore et dolore magna aliqua.
\end{frame}

% Listas
\subsection{Listas}

\begin{frame}
    \frametitle{Lista}
    \begin{itemize}
    \item Point A
    \item Point B
    \begin{itemize}
        \item part 1
        \item part 2
    \end{itemize}
    \item Point C
    \item Point D
    \end{itemize}
\end{frame}

\begin{frame}
    \frametitle{Lista numerada}
    \begin{enumerate}[I]
    \item Point A
    \item Point B
        \begin{enumerate}[i]
        \item part 1
        \item part 2
        \end{enumerate}
    \item Point C
    \item Point D
    \end{enumerate}    
\end{frame}

% Columns
\subsection{Columnas}

\begin{frame}
    \frametitle{Columnas}
    \begin{columns}
    \column{0.4\textwidth}
        Lorem ipsum dolor sit amet, consectetur adipisicing elit, sed do eiusmod tempor incididunt ut labore et dolore magna aliqua.
    \column{0.4\textwidth}
        Lorem ipsum dolor sit amet, consectetur adipisicing elit, sed do eiusmod tempor incididunt ut labore et dolore magna aliqua.
    \end{columns}
\end{frame}

% Bloques y código
\subsection{Bloques y código}

\begin{frame}[fragile] % No olvides pasarle el comando fragile a la diapositiva si quieres usar verbatim para mostrar código
    \frametitle{Bloques y código}
    \begin{block}{Este código no tiene sentido} % Puede dársele un título al bloque o dejar el texto entre las llaves en blanco
        % Verbatim respeta las tabulaciones. No las incluyas por respetar la indexación del documento si no quieres que se incluyan también en el documento final
        \begin{verbatim} 
for i in range(1, 5):
  print i
else:
  print "The for loop is over"
        \end{verbatim}
    \end{block}
\end{frame}

% Imágenes
\subsection{Imágenes}

\begin{frame}
    \frametitle{Imágenes}
    \begin{figure}
        \includegraphics[scale=0.5]{img/logo1.png}
        \caption{Logo del GUL}
    \end{figure}
    Lorem ipsum dolor sit amet, consectetur adipisicing elit, sed do eiusmod tempor incididunt ut labore et dolore magna aliqua.
\end{frame}

% Tablas
\subsection{Tablas}

\begin{frame}
    \frametitle{Tablas}
    \begin{table}
        \begin{tabular}{l | c | c | c | c }
            Name & Swim & Cycle & Run & Total \\
            \hline \hline
            John T & 13:04 & 24:15 & 18:34 & 55:53 \\ 
            Norman P & 8:00 & 22:45 & 23:02 & 53:47\\
            Alex K & 14:00 & 28:00 & n/a & n/a\\
            Sarah H & 9:22 & 21:10 & 24:03 & 54:35 
        \end{tabular}
        \caption{Tiempos}
    \end{table}
\end{frame}

% Varios elementos
\subsection{Mezcla de elementos}

\begin{frame}[fragile]
    \frametitle{Imágenes y bloques en columnas}
    \begin{columns}
        \column{0.4\textwidth}
            \begin{figure}
                \includegraphics[scale=0.5]{img/logo1.png}
                \caption{Logo del GUL}
            \end{figure}
        \column{0.5\textwidth}
            \begin{block}{Pueden unirse varios elementos en una misma diapositiva} % Puede dársele un título al bloque o dejar el texto entre las llaves en blanco
                \begin{verbatim}
for i in range(1, 5):
  print i
else:
  print "The for loop is over"
                \end{verbatim}
            \end{block}
    \end{columns}
\end{frame}

\end{document}
