\documentclass[aspectratio=43]{beamer}

\usetheme{simple}

\usepackage{lmodern}
\usepackage[scale=2]{ccicons}

\usepackage[utf8]{inputenc} % No olvides utilizar UTF8, especialmente si tu presentación está en castellano


\def\edicion{XXXV} % Edición de las jornadas
\def\fecha{Abril 2023} % Fecha de las jornadas

\title{Tu titulo} % Modifica el título de la presentación a tu gusto
\subtitle{Tu subtitulo} % Incluye un subtítulo si lo ves necesario, o bórralo
\author{Tu nombre} % Cambia el autor de la presentación
\twitter{TuTwitter} % Pon tu Twitter o déjalo en blanco

% Esto no lo cambies
\institute{\edicion \ Jornadas Técnicas del GUL}
\date{\fecha}


% Página principal (no tocar)
\titlegraphic{img/logo1.png}

\begin{document}

{
    \setbeamertemplate{footline}{}
    \begin{frame}
        \titlepage
    \end{frame}
}
\addtocounter{framenumber}{-1}

%\setwatermark{\includegraphics[height=8cm]{img/logo1.png}} % Marca de agua (opcional)

% ------------------------------------------------------------------------------------------
% La presentación empieza aquí. La primera diapositiva puede dejarse tal cual o sustituirse.
% ------------------------------------------------------------------------------------------

\begin{frame}
    \frametitle{Tabla de contenidos} % Puedes ponerle un título más chulo a la diapositiva con el índice de contenidos o dejar éste
    \tableofcontents % Para poblar la tabla de contenidos debes utilizar \section y \subsection apropiadamente (ver demo)
\end{frame}

\begin{frame}
    \frametitle{}

\end{frame}


\end{document}
